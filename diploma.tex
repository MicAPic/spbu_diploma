% ОБЯЗАТЕЛЬНО ИМЕННО ТАКОЙ documentclass!
% (Основной кегль = 14pt, поэтому необходим extsizes)
% Формат, разумеется, А4
% article потому что стандарт не подразумевает разделов
% Глава = section, Параграф = subsection
% (понятия "глава" и "параграф" из документа, описывающего диплом)
\documentclass[a4paper,article,14pt]{extarticle}

% Подключаем главный пакет со всем необходимым
\usepackage{spbudiploma_tempora}

% Пакеты по желанию (самые распространенные)
% Хитрые мат. символы
\usepackage{euscript}
% Таблицы
\usepackage{longtable}
\usepackage{makecell}
% Картинки (можно встявлять даже pdf)
\usepackage[pdftex]{graphicx}

\usepackage{amsthm,amssymb, amsmath}
\usepackage{textcomp}


\begin{document}

% Титульник в файле titlepage.tex
\input{parts/titlepage.tex}

% Содержание
\tableofcontents
\pagebreak

\specialsection{Введение}

\specialsubsection{Актуальность работы}
В настоящее время разработка компьютерных игр (\textit{геймдев}, \textit{gamedev}) является быстроразвивающейся индустрией. При этом научные исследования последних лет показывают, что методологии и достижения в разработке компьютерных игр находят применение не только в сфере развлечений, но и в:
\begin{itemize}
	\item[--] медицине (терапия расстройств аутистического спектра (РАС), детского церебрального паралича (ДЦП)) \cite{lopes2018games,hassan2021serious};
	\item[--] образовании (обучение программированию и другим дисциплинам) \cite{vahldick2020blocks,mayer2019computer};
	\item[--] бизнесе (геймификация корпоративных систем) \cite{augustin2016we}.
\end{itemize}

При разработке игр команды нередко сталкиваются с рядом глобальных проблем. К ним можно отнести как технические ошибки (недостаточная оптимизация, баги), так и ошибки в дизайне (неясное видение дизайна игры, отсутствие <<фактора удовольствия>>) \cite{politowski2021game}.

Все большее количество проблем исследователи пытаются разрешить с помощью использования искусственного интеллекта. Основными сферами применения ИИ являются создание правдоподобных неигровых персонажей (\textit{non-playable character}, \textit{NPC}), процедурная генерация контента, моделирование опыта игроков, помощь с геймдизайном \cite{xia2020recent}.

Одним из факторов возникновения проблем является ручное тестирование, превалирующее в разработке видеоигр \cite{santos2018computer}, которое не всегда позволяет выявить возможные ошибки. Оно полностью полагается на \textit{ad-hoc} мышление игровых тестировщиков, играющих в одну и ту же игру снова и снова. Все чаще игры выходят с ошибками, которые разработчики вынуждены исправлять в последующих обновлениях \cite{truelove2021we}. В этой связи актуальной задачей является автоматизирование тестирования игрового процесса, отличающегося от <<традиционного>> процесса разработки ПО и слабо поддающегося устоявшимся методам автоматизации \cite{politowski2021game,murphy2014cowboys}. Частичная или полная автоматизация позволила бы специалистам сконцентрироваться на других особенностях игр, таких как баланс, логика и фактор удовольствия.

Перспективным инструментом автоматизации является обучение с подкреплением (\textit{reinforcement learning}, \textit{RL}) на основе нейронных сетей. RL успешно показывает себя в хорошо понимаемых, воспроизводимых процессах. Кроме того, алгоритмы обучения с подкреплением позволяют создавать агентов, которые соответствуют разным стилям пользовательского поведения.

Данная работа посвящена попытке интегрирования этой методологии в процесс разработки игр.

\specialsubsection{Цель работы}
Целью настоящей работы является тестирование игрового процесса на основе обучения с подкреплением при разработке компьютерных игр с целью повышения их качества, а также минимизации времени и ресурсов, затрачиваемых на тестирование разрабатываемых видеоигровых продуктов.

\specialsubsection{Задачи работы}
Для выполнения вышеуказанной цели были поставлены следующие задачи:
\begin{enumerate}
	\item Обзор существующей литературы, посвященной тестированию игрового процесса и его автоматизации на основе обучения с подкреплением;
	\item Обзор современных методов обучения с подкреплением, использующихся в \textit{gamedev};
	\item Проведение обзора метрик качества методов на основе обучения с подкреплением;
	\item Проектирование и реализация архитектуры игровой среды для эмуляции игрового процесса;
	\item Разработка и оценка качества метода тестирования игрового процесса на основе моделей обучения с подкреплением;
	\item Тестирование и апробация разработанного решения.
\end{enumerate}

\specialsubsection{Практическая значимость работы}
Данное исследование призвано помочь специалистам в области разработки игр эффективно заменить часть ручного тестирования, сократив время, затрачиваемое на тестирование, а также, в конечном счете, существенно повысить качество игрового процесса.

Таким образом, данная дипломная работа может внести значительный вклад в современную индустрию разработки компьютерных игр, а также в области науки, использующие достижения и технологии геймдева.

%
%Проверяем как у нас работают сокращения, обозначения и определения "---
%MAX, 
%\Abbrev{MAX}{Maximum ""--- максимальное значение параметра}
%API 
%\Abbrev{API}{application programming interface ""--- внешний интерфейс взаимодействия с приложением}
%с обратным прокси.
%\Define{Обратный прокси}{тип прокси-сервера, который ретранслирует}





\section{Ненастоящее введение}
\subsection{Мотивация}
И нет сомнений, что действия представителей оппозиции ограничены исключительно образом мышления. Значимость этих проблем настолько очевидна, что дальнейшее развитие различных форм деятельности, а также свежий взгляд на привычные вещи - безусловно открывает новые горизонты для модели развития. Ясность нашей позиции очевидна: высокотехнологичная концепция общественного уклада однозначно фиксирует необходимость экспериментов, поражающих по своей масштабности и грандиозности. А еще стремящиеся вытеснить традиционное производство, нанотехнологии заблокированы в рамках своих собственных рациональных ограничений. Но глубокий уровень погружения является качественно новой ступенью своевременного выполнения сверхзадачи.

Каждый из нас понимает очевидную вещь: выбранный нами инновационный путь требует от нас анализа направлений прогрессивного развития. Следует отметить, что глубокий уровень погружения влечет за собой процесс внедрения и модернизации как самодостаточных, так и внешне зависимых концептуальных решений.

Ненумерованная формула:

\begin{equation}
    \begin{pmatrix} \dot{\varphi}\\ \dot{\theta} \\ \dot{\psi} \end{pmatrix}
    = \begin{pmatrix}
        cos(\theta)cos(\psi) & -sin(\psi) & 0 \\
        cos(\theta)sin(\psi) & cos(\psi)  & 0 \\
        -sin(\theta)         & 0         &  1
    \end{pmatrix}^{-1}
    \begin{pmatrix} \omega_x\\ \omega_y \\ \omega_z \end{pmatrix}.
\end{equation}

Нумерованная формула:

\begin{equation}
    i^2 = -1.
    \label{eq:my_ref}
\end{equation}

Тест ссылки на формулу \ref{eq:my_ref}.

Принимая во внимание показатели успешности, разбавленное изрядной долей эмпатии, рациональное мышление представляет собой интересный эксперимент проверки стандартных подходов. Равным образом, существующая теория напрямую зависит от кластеризации усилий! Имеется спорная точка зрения, гласящая примерно следующее: реплицированные с зарубежных источников, современные исследования подвергнуты целой серии независимых исследований. Высокий уровень вовлечения представителей целевой аудитории является четким доказательством простого факта: глубокий уровень погружения выявляет срочную потребность модели развития.

\subsection{Постановка задачи}

Безусловно, дальнейшее развитие различных форм деятельности способствует подготовке и реализации первоочередных требований. Современные технологии достигли такого уровня, что современная методология разработки однозначно фиксирует необходимость вывода текущих активов. В рамках спецификации современных стандартов, базовые сценарии поведения пользователей, инициированные исключительно синтетически, подвергнуты целой серии независимых исследований. Безусловно, дальнейшее развитие различных форм деятельности позволяет выполнить важные задания по разработке существующих финансовых и административных условий. Не следует, однако, забывать, что постоянный количественный рост и сфера нашей активности, а также свежий взгляд на привычные вещи - безусловно открывает новые горизонты для приоритизации разума над эмоциями. Постоянное информационно-пропагандистское обеспечение нашей деятельности играет важную роль в формировании позиций, занимаемых участниками в отношении поставленных задач.

Для современного мира разбавленное изрядной долей эмпатии, рациональное мышление играет определяющее значение для стандартных подходов. Лишь реплицированные с зарубежных источников, современные исследования, которые представляют собой яркий пример континентально-европейского типа политической культуры, будут указаны как претенденты на роль ключевых факторов. Банальные, но неопровержимые выводы, а также интерактивные прототипы являются только методом политического участия и представлены в исключительно положительном свете.

\subsection{Доступные программные средства}

Значимость этих проблем настолько очевидна, что начало повседневной работы по формированию позиции представляет собой интересный эксперимент проверки прогресса профессионального сообщества. С другой стороны, высокотехнологичная концепция общественного уклада требует определения и уточнения направлений прогрессивного развития.


Ниже тестируется очень большая таблица на несколько страниц

\begin{center}
    \begin{longtable}{|p{2cm}|p{3cm}|p{7cm}|p{3cm}|}
    \caption{Заголовок таблицы}\\
    \hline
    1 & 2 & 3 & 4\\ 
    \hline 
    2 & 2 & 3 & 4\\
    \hline
    3 & 2 & 3 & 4\\
    \hline
    4 & 2 & 3 & 4\\
    \hline
    5 & 2 & 3 & 4\\
    \hline
    6 & 2 & 3 & 4\\
    \hline
    7 & 2 & 3 & 4\\
    \hline
    8 & 2 & 3 & 4\\
    \hline
    9 & 2 & 3 & 4\\
    \hline
    10 & 2 & 3 & 4\\
    \hline
    
    
    \end{longtable}
\end{center}


А также тестируется счетчик таблиц, жирные и двойные линии.

\begin{center}
    \begin{longtable}{|p{2cm}||p{3cm}|p{7cm}|p{3cm}|}
    \caption{Заголовок таблицы нумер 2}\\
    \hline
    1 & 2 & 3 & 4\\ 
    \hline
    2 & 2 & 3 & 4\\
    \hline
    3 & 2 & очень жирная ячейка \par с переносом (работаеттт!) & 4\\
    \hline
    4 & 2 & 3 & 4\\
    \hline
    5 & 2 & 3 & 4\\
    \hline
    6 & 2 & 3 & 4\\
    \hline
    7 & 2 & 3 & 4\\
    \hline
    8 & 2 & 3 & 4\\
    \hline
    9 & 2 & 3 & 4\\
    \hline
    10 & 2 & 3 & 4\\
    \hline
    
    
    \end{longtable}
\end{center}


\subsection{Полученные результаты} 

Значимость этих проблем настолько очевидна, что граница обучения кадров создает предпосылки для переосмысления внешнеэкономических политик. Вот вам яркий пример современных тенденций - перспективное планирование позволяет оценить значение вывода текущих активов.

\section{Основная часть раз}
Равным образом, социально-экономическое развитие не дает нам иного выбора, кроме определения вывода текущих активов. Высокий уровень вовлечения представителей целевой аудитории является четким доказательством простого факта: семантический разбор внешних противодействий позволяет оценить значение новых предложений.

Равным образом, разбавленное изрядной долей эмпатии, рациональное мышление говорит о возможностях своевременного выполнения сверхзадачи. Высокий уровень вовлечения представителей целевой аудитории является четким доказательством простого факта: выбранный нами инновационный путь предоставляет широкие возможности для системы массового участия. Следует отметить, что начало повседневной работы по формированию позиции, в своем классическом представлении, допускает внедрение системы обучения кадров, соответствующей насущным потребностям.

\pagebreak
\section{Основная часть два: Теория}

\section{Основная часть два: Детали реализации}
\subsection{Расчётная часть}

\section{Анализ экспериментов.}
\begin{figure}[ht]
\begin{center}
\scalebox{0.4}{
   \includegraphics{images/graph.jpg}
}

\caption{
\label{graph-fig}
     Линейные функции.}
\end {center}
\end {figure}
Ссылаемся на график ~\ref{graph-fig}.
%Ссылка на статью: \cite{voc}, \cite{vo2}

\specialsection{Выводы}
Жизнь --- тлен.
\pagebreak

\specialsection{Заключение}

С другой стороны, консультация с широким активом обеспечивает актуальность форм воздействия. Следует отметить, что выбранный нами инновационный путь создает необходимость включения в производственный план целого ряда внеочередных мероприятий с учетом комплекса благоприятных перспектив. В частности, реализация намеченных плановых заданий влечет за собой процесс внедрения и модернизации поэтапного и последовательного развития общества. В частности, новая модель организационной деятельности способствует подготовке и реализации стандартных подходов и тому подобных экспериментов.

% Библиография в cpsconf стиле
% Аргумент {1} ниже включает переопределенный стиль с выравниванием слева
\begin{thebibliography}{1}
	
%%%%%%%%%%%%%%% Введение. Область применения
\bibitem{lopes2018games} Lopes~S. et al. Games used with serious purposes: a systematic review of interventions in patients with cerebral palsy // Frontiers in psychology. 2018. Т.~9. С.~312205.
\bibitem{hassan2021serious} Hassan~A., Pinkwart~N., Shafi~M. Serious Games to Improve Social and Emotional Intelligence in Children With Autism // Entertainment Computing. 2021. Т.~38. С.~100417.
\bibitem{vahldick2020blocks} Vahldick~A. et al. A blocks-based serious game to support introductory computer programming in undergraduate education // Computers in Human Behavior Reports. 2020. Т.~2. С.~100037.
\bibitem{mayer2019computer} Mayer~R.E. Computer Games in Education // Annual review of psychology. 2019. Т.~70. С.~531--549.
\bibitem{augustin2016we} Augustin~K. et al. Are We Playing Yet? A Review of Gamified Enterprise Systems // PACIS 2016 Proceedings. 2016.

%%%%%%%%%%%%%%% Введение. Обзорные статьи
\bibitem{xia2020recent} Xia~B., Ye~X., Abuassba~A.O.M. Recent Research on AI in Games // 2020 International Wireless Communications and Mobile Computing (IWCMC). 2020. С.~505--510.
\bibitem{santos2018computer} Santos~R.E.S. et al. Computer Games Are Serious Business and So Is Their Quality: Particularities of Software Testing in Game Development From the Perspective of Practitioners // Proceedings of the 12th ACM/IEEE International Symposium on Empirical Software Engineering and Measurement. 2018. С.~1--10.
\bibitem{truelove2021we} Truelove~A., de Almeida~E.S., Ahmed~I. We'll Fix It in Post: What Do Bug Fixes in Video Game Update Notes Tell Us? // 2021 IEEE/ACM 43rd International Conference on Software Engineering (ICSE). IEEE, 2021. С.~736--747.
\bibitem{politowski2021game} Politowski~C. et al. Game industry problems: An extensive analysis of the gray literature // Information and Software Technology. 2021. Т.~134. С.~106538.
\bibitem{murphy2014cowboys} Murphy-Hill~E., Zimmermann~T., Nagappan~N. Cowboys, ankle sprains, and keepers of quality: How is video game development different from software development? // Proceedings of the 36th International Conference on Software Engineering. 2014. С.~1-11.

\end{thebibliography}
\end{document}