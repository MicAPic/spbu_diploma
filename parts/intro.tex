\specialsection{Введение}

\specialsubsection{Актуальность работы}
В настоящее время разработка компьютерных игр (\textit{геймдев}, \textit{gamedev}) является быстроразвивающейся индустрией. При этом научные исследования последних лет показывают, что методологии и достижения в разработке компьютерных игр находят применение не только в сфере развлечений, но и в:
\begin{itemize}
	\item[--] медицине (терапия расстройств аутистического спектра (РАС), детского церебрального паралича (ДЦП)) \cite{lopes2018games,hassan2021serious};
	\item[--] образовании (обучение программированию и другим дисциплинам) \cite{vahldick2020blocks,mayer2019computer};
	\item[--] бизнесе (геймификация корпоративных систем) \cite{augustin2016we}.
\end{itemize}

При разработке игр команды нередко сталкиваются с рядом глобальных проблем. К ним можно отнести как технические ошибки (недостаточная оптимизация, баги), так и ошибки в дизайне (неясное видение дизайна игры, отсутствие <<фактора удовольствия>>) \cite{politowski2021game}.

Все большее количество проблем исследователи пытаются разрешить с помощью использования искусственного интеллекта. Основными сферами применения ИИ являются создание правдоподобных неигровых персонажей (\textit{non-playable character}, \textit{NPC}), процедурная генерация контента, моделирование опыта игроков, помощь с геймдизайном \cite{xia2020recent}.

Одним из факторов возникновения проблем является ручное тестирование, превалирующее в разработке видеоигр \cite{santos2018computer}, которое не всегда позволяет выявить возможные ошибки. Оно полностью полагается на \textit{ad-hoc} мышление игровых тестировщиков, играющих в одну и ту же игру снова и снова. Все чаще игры выходят с ошибками, которые разработчики вынуждены исправлять в последующих обновлениях \cite{truelove2021we}. В этой связи актуальной задачей является автоматизирование тестирования игрового процесса, отличающегося от <<традиционного>> процесса разработки ПО и слабо поддающегося устоявшимся методам автоматизации \cite{politowski2021game,murphy2014cowboys}. Частичная или полная автоматизация позволила бы специалистам сконцентрироваться на других особенностях игр, таких как баланс, логика и фактор удовольствия.

Перспективным инструментом автоматизации является обучение с подкреплением (\textit{reinforcement learning}, \textit{RL}) на основе нейронных сетей. RL успешно показывает себя в хорошо понимаемых, воспроизводимых процессах. Кроме того, алгоритмы обучения с подкреплением позволяют создавать агентов, которые соответствуют разным стилям пользовательского поведения.

Данная работа посвящена попытке интегрирования этой методологии в процесс разработки игр.

\specialsubsection{Цель работы}
Целью настоящей работы является тестирование игрового процесса на основе обучения с подкреплением при разработке компьютерных игр с целью повышения их качества, а также минимизации времени и ресурсов, затрачиваемых на тестирование разрабатываемых видеоигровых продуктов.

\specialsubsection{Задачи работы}
Для выполнения вышеуказанной цели были поставлены следующие задачи:
\begin{enumerate}
	\item Обзор существующей литературы, посвященной тестированию игрового процесса и его автоматизации на основе обучения с подкреплением;
	\item Обзор современных методов обучения с подкреплением, использующихся в \textit{gamedev};
	\item Проведение обзора метрик качества методов на основе обучения с подкреплением;
	\item Проектирование и реализация архитектуры игровой среды для эмуляции игрового процесса;
	\item Разработка и оценка качества метода тестирования игрового процесса на основе моделей обучения с подкреплением;
	\item Тестирование и апробация разработанного решения.
\end{enumerate}

\specialsubsection{Практическая значимость работы}
Данное исследование призвано помочь специалистам в области разработки игр эффективно заменить часть ручного тестирования, сократив время, затрачиваемое на тестирование, а также, в конечном счете, существенно повысить качество игрового процесса.

Таким образом, данная дипломная работа может внести значительный вклад в современную индустрию разработки компьютерных игр, а также в области науки, использующие достижения и технологии геймдева.

%
%Проверяем как у нас работают сокращения, обозначения и определения "---
%MAX, 
%\Abbrev{MAX}{Maximum ""--- максимальное значение параметра}
%API 
%\Abbrev{API}{application programming interface ""--- внешний интерфейс взаимодействия с приложением}
%с обратным прокси.
%\Define{Обратный прокси}{тип прокси-сервера, который ретранслирует}



